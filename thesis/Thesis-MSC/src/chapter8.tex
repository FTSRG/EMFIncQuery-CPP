\chapter{Conclusion}

My job was to create an application capable of generating code from an EMF
model, able to declare queries over said model and run these queries over the
\CPP{} instances of the generated classes. I also had to test it's scalability
and efficiency.

I managed to finish the following tasks:

\begin{itemize}
  \item Generating code from EMF model
  \item Generate search plan from the queries written in EMF-IncQuery Pattern
  Language
  \item Execution of the search plan over \CPP{} instances
  \item Scaleability and performance testing
\end{itemize}

During my work I did not manage to implement every feature of the original
EMF-IncQuery engine, as the handling of constants, evaluatuon and check
expressions and negative pattern application is still missing. The negative
pattern applications could be implemented by generating a version of the
negatively applied pattern where all the parameters are bound, calling this and
checking if there are any matches as a search operation. The check and eval
operations are more problematic, as they would require the modification of the
pattern language. The EMF-IncQuery pattern language allows the writing of check
expression using the Xbase language, which allows the usage of Java libraries.
Translating this to Java would be unreasonable amount of work, thus restricting
the usable expression is the better solution.

Throughout development, the fact that the Java local search engine was
still being implemented slowed down my job a significantly. There were several
occations when a bug in the search plan generation would break the \CPP{}
generated code, components did not behave as documented or the Java API was not
planned with external contributions in mind, resulting in temporary hacks and
workarounds.

There are still a lot of work left to be done on the resulting program. The
generated code should is currently not tested, this resulted in many unexpected
bugs when encountering with specific edge cases. Another possible improvement
would be the implementation of a multithreaded version of the search plan
execution. This could result in significant performance gains in the case of
larger models and more complex queries. In addition, it would be worthwile to
check different ways of solving the \CPP{} related problems mentioned in section
\sectref{CppSpecificProblems}. For example it might be a reasonable restraint
for all of the model classes to inherit from a common base type, as \CPP{}
allows for multiple inheritance.

All in all, I believe the resulting program, while definitely not production
ready, proved that it might be worthwhile to put further research and
development in the \CPP{} implementation of search plan execution, as the
measured performance was significantly better in most cases.
