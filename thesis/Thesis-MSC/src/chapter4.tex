\chapter{Elaboration}

The aim of this chapter is to provide an example of the full workflow using the
created program throughout a realistic scenario. This section will show how to
use a predefined metamodel to write queries, generate model and query code, and
describes the usage of the generated code with a simple code example. This
example will use the previously defined \emph{school} metamodel
(\figref{School_Metamodel}). The example assumes an already created
eclipse modeling project with the metamodel inside it.

\section{Writing queries}

The first step of the workflow is determining the required queries and writing
them. The scenario is that the \CPP{} program requires a query to determine the
school a student goes to in an efficient manner. To do this, an \emph{.eiq} file
must be created in the project on the source path. The contents of the
\emph{.eiq} will be described in listing \listref{example_school_eiq}.

\begin{lstlisting}[frame=single,float=!ht,language=IQPL,
label=listing:example_school_eiq, caption=The \emph{.eiq} file for the example
project] 
package hu.bme.mit.cpp.localsearch.school.query

import "http://school.ecore"

@Bind(parameters={school})
pattern teachersOfSchool(teacher, school) {
	School.teachers(school, teacher);
}
\end{lstlisting}

The \emph{.eiq} file contains a package declaration, an import to the metamodel
and a single pattern defining the required query. The query returns all
school-teacher pairs where the teacher teaches at the given school. The
\emph{@Bind} annotation defines that the student is a bound parameter, which
means it can be specified from the code so that the query will only search for
the school.

\section{Code generation}

The next step is generating the model and query code. The model code can be
generated via right clicking the \emph{.ecore} file describing the metamodel
and using the \emph{EMF Generator -> Generate \CPP{} files} option. This generates
the \CPP{} classes from the metamodel in the \emph{cpp-gen} folder. The query
code can be generated in similar fashion, by right clicking the \emph{.eiq} file
and selecting the appropriate option from the \emph{Generate query code\ldots}
menu. The available options are either the runtime based or the iterator based
implementation. For more on the differences of these generated code styles see
section \sectref{generated_code_and_api}.

Generating the runtime code results in three files, \emph{schoolQueries.h},
\emph{TeachersOfSchoolMatch.h} and \emph{TeachersOfSchoolFrame.h}. Listing
\listref{tos_frame} shows part of the content of \emph{TeachersOfSchoolFrame.h}.
Some trivial code like includes, namespace declarations and header guard are
omitted.

\begin{lstlisting}[frame=single,float=!ht,language=C++,
label=listing:tos_frame, caption=The contents of \emph{TeachersOfSchoolFrame.h}] 
struct TeachersOfSchoolFrame {
	Teacher* _0;
	School* _1;
};
\end{lstlisting}

The header file contains a declaration of a struct
(\emph{TeachersOfSchoolFrame}), which contains all the pattern variables used
during query execution. In this case, the variables match the patterns output
parameters, but this is not alway the case, as additional variables might be
necessary.

The generated match file is slightly more complex as shown by listing
\listref{tos_match}.

\begin{lstlisting}[frame=single,float=!ht,language=C++,
label=listing:tos_match, caption=The contents of \emph{TeachersOfSchoolMatch.h}] 
struct TeachersOfSchoolMatch {	
	School* school;
	Teacher* teacher;
	
	bool operator==(const TeachersOfSchoolMatch& other) const {
		return 
			school == other.school &&
			teacher == other.teacher;
	}	
};

namespace std {
template<> struct hash<TeachersOfSchoolMatch> {
	unsigned operator()(const TeachersOfSchoolMatch& match) const {
		return 
			std::hash<decltype(match.school)>()(match.school)^
			std::hash<decltype(match.teacher)>()(match.teacher);
	}
};		
}
\end{lstlisting}

The match file contains a struct with pointers to the objects which are part of
the match. In this case with the names the user defined in the query. In
addition, an equality operator and a hash function is also defined for the
matches. These are used to determine the matches uniqueness, so that the same
match does not get returned multiple times in a single query.

The third file, \emph{schoolQueries.h} defines the class which contains the
methods for query execution. As shown in listing \listref{tos_query_ctr}, its
constructor initializes the \emph{ClassHelper} with the type hierarchy
information from the metamodel. This is used in order to help with the type
checking.

\begin{lstlisting}[frame=single,float=!ht,language=C++,
label=listing:tos_query_ctr, caption=The contents of
\emph{schoolQueries.h}] 
schoolQueries() {
	_classHelper = ClassHelper::builder()
		.forClass(Course::type_id).noSuper()
		.forClass(School::type_id).noSuper()
		.forClass(SchoolClass::type_id).noSuper()
		.forClass(SpecialisationCourse::type_id).setSuper(Course::type_id)
		.forClass(Student::type_id).noSuper()
		.forClass(Teacher::type_id).noSuper()
		.forClass(Year::type_id).noSuper()
		.forClass(LimitedCapacityCourse::type_id).setSuper(Course::type_id)
		.build();
}
\end{lstlisting}

From the \emph{teachersOfSchool} pattern four methods generate, two used if no
binding is done and two if school is bound. For both bound and unbound queries
there are versions which return all matches or a single arbitrary match. Listing
\listref{tos_query_gaub} shows the unbound method which returns all matches.
First, it creates the collection for the matches and the search context, than an
empty search plan. Next, it assembles the search plan. The plan consists of
two operations in this case. The first operation simply iterates through all
the schools and store them in the first slot of the frame, one at a time. The
second operation navigates from the actually stored school, through the teachers
association to all the teachers of the school and also stores them one at a time
in the zeroth slot of the frame. Following the creation of the search plan, an
executor gets instantiated. The last step is the iteration of the matches using
the executors iterators, which returns the frame when a match is found. Each
matching frame gets transformed into an actual match object and the match gets
added to the matches collection. The query than returns the resulting set of
matches which should contain all matching objects from the model.



\begin{lstlisting}[frame=single,float=!ht,language=C++,
label=listing:tos_query_gaub, caption=The contents of
\emph{schoolQueries.h}] 
std::unordered_set<TeachersOfSchoolMatch>
get_all_teachers_of_school() { 
	std::unordered_set<TeachersOfSchoolMatch> matches;
	ISearchContext isc(_classHelper);
	
	SearchPlan< TeachersOfSchoolFrame> sp;
	
	sp.add_operation(create_IterateOverInstances(
						&TeachersOfSchoolFrame::_1,
						School::type_id));
	sp.add_operation(create_NavigateMultiAssociation(
						&TeachersOfSchoolFrame::_1,
						&TeachersOfSchoolFrame::_0, 
						&School::teachers)); 
	
	SearchPlanExecutor<TeachersOfSchoolFrame> exec(sp, isc);
	SearchPlanExecutor<TeachersOfSchoolFrame>::iterator it;
	
	for(it = exec.begin(); it != exec.end(); it++) {
		TeachersOfSchoolMatch match;	
		match.teacher = static_cast<::school::school::Teacher*>((*it)._0);
		match.school = static_cast<::school::school::School*>((*it)._1);
		matches.insert(match);
	}
	return matches;
}
\end{lstlisting}

Listing \listref{tos_query_gab} shows the implementation of the bound version.
The code is mostly similar to the previous version. The first difference is that
the method has a parameter now, which is the bound parameter. Because of the
bound parameter, the search plan is slightly different. Since the school is
already provided via method argument, it is not necessary to iterate over all
the schools, only the navigation step is necessary. As such, the iteration
operation is missing from the search plan. The next difference is the creation
of an empty frame and its initialization. 

\begin{lstlisting}[frame=single,float=!ht,language=C++,
label=listing:tos_query_gab, caption=The contents of
\emph{schoolQueries.h}]
std::unordered_set<TeachersOfSchoolMatch>
get_all_teachers_of_school(School* school) { 
	std::unordered_set<TeachersOfSchoolMatch> matches;
	ISearchContext isc(_classHelper);
	
	SearchPlan<TeachersOfSchoolFrame> sp;
	
	sp.add_operation(create_NavigateMultiAssociation(
					&TeachersOfSchoolFrame::_1,
					&TeachersOfSchoolFrame::_0,
					&School::teachers));
	
	SearchPlanExecutor<TeachersOfSchoolFrame> exec(sp, isc);
	TeachersOfSchoolFrame frame;
	
	frame._1 = school;
	
	while(exec.execute(frame)) {
		TeachersOfSchoolMatch match;	
		match.teacher = static_cast<::school::school::Teacher*>(frame._0);
		match.school = static_cast<::school::school::School*>(frame._1);
		matches.insert(match);
	}
	
	return matches;
}
\end{lstlisting}

The frame has to be initialized with
the provided bound parameters, so the query execution knows of the bound values.
As the frame has to be prepared, the execution is done via the executors execute
method, which requires a frame as a parameter which will be used during
execution. The creation of the match is the same as before.

\section{Using the generated code}

The final step is the usage of the generated code. The code generator is
designed so that the \emph{cpp-gen} folder can be used as a project folder. The
model generator also generates a \emph{makefile} which can be used to build the
project with \emph{GNU Make}. The \emph{makefile} should work out of the box
most of the time, but if the runtime library is not installed inside the default
library and include paths for the system or is not next to the original modeling
project, some modifications might be required. The makefile defines the
\emph{CXXFLAGS} and \emph{LIBPATH} variables, these have to be modified to use
the proper runtime library path. If the makefile is properly configured, the
easiest way to use the generated code is by creating a \emph{main.cpp} file in
the \emph{cpp-gen} folder containing the user code. In this example the contents
of the \emph{main.cpp} file can be seen in listing \listref{example_school_cpp}.

\begin{lstlisting}[frame=single,float=!ht,language=C++,
label=listing:example_school_cpp, caption=Fragment of the \emph{main.cpp}
file for the example project]

#include "school/school_def.h"
#include "Localsearch/school/schoolQueries.h"
// additional includes...

Student* initialize_model(); // implementation omitted

int main(int argc, char** args) {
	const Student* student = initialize_model();
	
	SchoolQueries engine;
	Optional<StudentsOfSchoolMatch> m = engine.get_one_students_of_school(student); 
	if(m.isPresent()) { 
		std::cout << "School of " 
				  << student.name << ": " 
				  << m.get().school.name <<	std::endl; 
	}
}
\end{lstlisting}

The example code uses \emph{\texttt{initialize\_model}} method to initialize the
model structure. It also returns the student whose school will be searched for.
The code then initializes the query engine and executes the query searching for
a single arbitrary school the student goes to. The returned value is an optional
of a match, which means the value might or might not be there. If there are no
matches to the query, then there is no result. Using the optional classes
\emph{isPresent()} method it is possible to determine whether a result was found or not.
If the result is present, then the code prints out the found school for the
student. The example code omits several include and using declaration and the
implementation of \emph{\texttt{initialize\_model}} method.

Building this code with running the \emph{make} command will result in an
executable which will initialize a model in the memory, execute a search on it
and print out the results.

